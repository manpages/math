 \documentclass[math.tex]{subfiles}
\begin{document}
\chapter{Вступление}

	\LARGE{
		$\cdot$ про ''Математику для Кипятильников''\\
		$\cdot$ история математики
	}\\

\pagebreak

%\chead{\small{Article 0 - Preface - $v0.0.1_{Rus}$}}
%\cfoot{-\thechapter-}

	\LARGE{Зачем это?}\\
	\normalsize
	
	Действительно, читатель может задаться вполне резонным вопросом —\ зачем в эпоху, когда доступна фактически любая книга по математике писать цикл статей про царицу наук? 
Кому эти статьи могут быть полезны и имеет ли смысл тратить время на их чтение?

	Отвечая на эти, безусловно правомерные вопросы, я начну с пары слов о том, как сам изучал математику.
Не пугайтесь, если названия областей математики, о которых я упоминаю в этой статье Вам не знакомы, как я напишу далее, серия ориентирована на широкий круг людей, так что, все важные детали будут подробно разъяснены.
	
	Появился интерес к этой науке у меня сравнительно поздно. Этому факту я обязан своему школьному учителю —\ Виктору Васильевичу Глухову. 
Подумать только, ведь если бы не он, я был бы совсем другим человеком, с другими интересами, другими знакомыми, другой жизнью. 
Ну это лирика, вернемся к теме. 
	
	Обучаясь школьной математике —\ пусть даже с гениальным преподавателем —\ многое остаятся за рамками изложения в классе, а следовательно за рамками понимания. 
К счастью, нам преподавали матем\-атику настолько последовательно и обстоятельно насколько это вообще возможно, нам старались давать ''честные'' определения большей части понятий, не вилять и не водить нас за нос при до\-ка\-за\-тельстве тех или иных теорем. 
Однако большýю часть природы и устройства математики приходилось постигать методом проб и ошибок (скажем, получая раз\-нос от пре\-под\-авателя за некорректное или неполное решение или наблюдая за ходом мысли учителя и почерпывая общие пред\-ставления о том, что можно делать в математике, а что нельзя).
	
	После того как я освоился с правилами игры, понял —\ пусть на интуитивном уровне —\ как математика устроена, я приступил к самостоятельному изучению математических теорий. 
Конечно, как не трудно до\-гадаться, первым, что я по\-старался изучить был математический анализ функций одного
переменного (в быту —\ матан). 
Давался он мне очень тяжело, теперь уже трудно сказать почему (сейчас математический анализ для меня что-то в духе арифметики: кажется, что знал всю жизнь и не ясно почему когда-то с ним были какие-то проблемы). 
Параллельно я стал изучать дискретную математику: логику по книге-брошюре И. Лаврова, теорию групп и алгебры по книге Ю. Шихановича.
Прочитав книгу Лаврова я влюбился в дискретную математику и стал интересоваться ей более серьезно, чем другими областями математики. 
Да и по-другому быть не могло, так как немногим ранее я послушал замечательный курс лекций по теории автоматов в ма\-те\-ма\-ти\-ческом лагере. 
Его читали Максим Александрович Ребров и Владимир Андреевич Редько. 
Эти люди произвели на меня огромное впечатление и помогли в фор\-мир\-овании математического кругозора как никто другой.
	
	Надо ли говорить, что пока я изучал дискретную математику, я —\ подобно слепому котенку —\ ''тыкался'' куда ни попадя. То возьму книгу по теории вер\-оятности, то почитаю про дифференциальные уравнения, то про линейную алгебру... 
Не то, чтобы от этого не было пользы или я чего-то не понимал, просто —\ и тут мы подходим к причине написания этой серии статей —\ у меня не было общей картинки. 
Как бы хорошо и подробно ни было написано введение в книгу по той или иной математической теории, я не мог уловить какое место она занимает во всей мозаике (исключение, пожалуй, предисловие к книге ''модальная логика'' Патрика Блэкбёрна).
	
	Идея, которая стоит за этой серией статей —\ дать читателю панорамную картину ряда математических теорий, а так же —\ что черезвычайно важно —\ показать связи между ними.
	Не последней задачей является обрисовать приложения той или иной теории, то есть какую непосредственную пользу принесет ее изучение.

	\vspace{1cm}
	\LARGE{Где это?}\\
	\normalsize
	
	Статьи я пишу в формате $TeX$ и они будут доступны в pdf где-то на сайте:\\ 
	\emph{http://memoricide.very.lv},\\ 
	в html в моём живом журнале:
	\\ \emph{http://manpages.livejournal.com},\\ 
	а $TeX$, в свою очередь, будет храниться на гитхабе:\\ 
	\emph{https://github.com/manpages/math}.
	
	
	Если у Вас есть или появятся какие-то рац. предложения, рац. дополнения или рац. коментарии по этим статьям, обязательно пишите на \emph{dva.traktorista@gmail.com}.

	\vspace{1cm}
	\LARGE{И все же... Что это?}\\
	\normalsize
	
	По состоянию на сегодня я планирую рассказать в своих статьях про математику с самых ее основ (лоника первого порядка, аксиоматика Цермело-Франкеля), далее затронуть ряд интересных математических понятий таких как алгебры, группы, кольца, алгебраические топологии. 
	
	В процессе изложения будут затронуты мат\-ематический анализ, пространства (в контексте топологии), возможно —\ теория доказательств.
	
	Финальной точкой серии статей будет взгляд на теорию категорий как на теорию, обощающую значительную часть описанного ранее.
	
	\vspace{1cm}
	\LARGE{Я ничего не понял. Это не для меня?}\\
	\normalsize
	
	Вовсе нет. 
При написании этой серии я буду ориентироваться на три основные, простите за каламбур, категории читателей —\ учащиеся старших классов, программисты, которые желают использовать деклар\-ативные языки программирования в полную силу и студенты высших учебных заведений, желающие получить пред\-ставление о тех теориях и понятиях, которые я буду описывать в своих статьях.
	
	Каждая статья будет содержать в себе предисловие, в котором будет описано отношение обсуждаемой теории или набора понятий к другим —\ уже упомянутым теориям и понятиям. 
Еще, в предисловии будет рассказано о том, какую пользу может вынести из статьи представитель той или
иной категории читателей.
При всей строгости и обстоятельности изложения, я буду стараться предварять фор\-мальные тексты и формулировки неформальными описаниями, а так же сопро\-вождать опре\-деления примерами и аналогиями.
Короче говоря, постараюсь сделать всё, чтобы текст легко воспринимался и был понятен любому читателю.

	Как уже должно было стать ясно, требований к читателю у этой серии статей нет. Разве что
желание разобраться в математике начиная с самых ее основ.
	
	Далее в этой —\ вводной —\ статье я приведу обзор эволюции математики от классической ее формы
к современной.
	
	\vspace{1cm}
	\LARGE{Версии документа}\\
	\normalsize
	
	Версии документам нужны для (по состоянию на сегодня —\ на случай) со\-вместной над ними работы.
Я искренне надеюсь на то, что рано или поздно люди станут не только комментировать и предлагать идеи по этой серии, но и редак\-тировать, вносить дополнения, а так же переводить ее на другие языки.
	
	Версии этого документа состоят из тройки чисел и индекса языка, например сейчас версия $v0.0.1_{Rus}$. 
Последняя цифра означает количество правок в текущую итерацию. 
Число посередине показывает сколько предложений и замечаний по тексту статьи внесено. 
Первое число показывает состояние документа, главную версию.

	Второе и третье числа локальны для каждой отдельно взятой статьи, а главная версия должна совпадать у всех статей серии.

	Главная версия будет выставлена в 1 когда все статьи серии, которые на данный момент мной задуманы, будут написаны.
	
	\vspace{1cm}
	\LARGE{Лицензия}\\
	\normalsize

	Attribution Share-alike (BY-SA):\\

	При распространении, дополнении и изменении серии лекций должен быть указано авторство про\-из\-ведения.
Производные произведения обязательно должны распространяться на условиях этой же лицензии.\\
	
	\emph{(Авторы каждой статьи указаны на титульной странице, в алфавитном порядке).}
	
	\vspace{1cm}
	\newpage
	\LARGE{История математики}
	\normalsize
	
	Введение в серию статей будет в известной степени описанием истории современной математики.
Легендарный преподаватель Латвийского Университета Рихардс Румниекс говорил, что знание истории черезвычайно важно для понимания природы вещей в той или иной области, но нужно знать не дату и время того или иного исторического события, а его причины и следствия.
Именно поэтому он начинал свой курс по архитектуре компьютерных систем с экскурса в историю вычислительных машин.
\end{document}